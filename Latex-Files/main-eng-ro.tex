% !TeX root = main.tex

\documentclass{article}
\usepackage{amsmath,amssymb,algorithm,algpseudocode,mathtools,calc}

\title{PID Controllers}
\author{AndreiDani}

\begin{document}
    \maketitle % -> Add title and author <- %

    \(\text{Hello World! This is where our journey starts. :DD}\)

    \bigbreak % --------------------------------------------------------------------------- %

    % ---> This is used to force intertext to limit extending (!!!) <--- %
    \setlength{\abovedisplayskip}{1pt} \setlength{\belowdisplayskip}{1pt}

    % ---> Global functions <--- %

    \makeatletter

    \newlength{\formulalinewidth}
    \newcommand{\measuremaxwidth}[1]{ 
        \settowidth{\@tempdima}{$#1$}
        \ifdim\@tempdima>\formulalinewidth
            \setlength{\formulalinewidth}{\@tempdima}
        \fi
    }

    \makeatother

    \bigbreak % -------------------------------------------->>>>>>>>> :) <<<<<<<<-------------------------------------------- %

    \settowidth{\formulalinewidth}{0pt}

    \measuremaxwidth{\gamma\, =\, \text{target angle}}
    \measuremaxwidth{\theta\, =\, \text{angle of the gyroscope}}
    \measuremaxwidth{E_{raw}(t)\, =\, \gamma - \theta}
    \measuremaxwidth{E_{raw}(t)\, =\, \text{error from the trajectory}}
    \measuremaxwidth{E(t)\, =\, E_{raw}(t)\, =\, \text{final error}}

    \addtolength{\formulalinewidth}{2em}

    \begin{align*}
        \gamma\, &=\, \text{target angle} \\
        \theta\, &=\, \text{angle of the gyroscope} \\
        % --------------------------------------------------------------------------- %        
        \intertext{\centering\rule[2.2pt]{\formulalinewidth}{0.4pt}} 
        % --------------------------------------------------------------------------- %
        E_{raw}(t)\, &=\, \gamma - \theta \\
        E_{raw}(t)\, &=\, \text{error from the trajectory} \\
        E(t)\, &=\, E_{raw}(t)\, =\, \text{final error} \\  
    \end{align*}

    \bigbreak % ------------------------------------------------------------------------ %
    \hrule % --------------------------------------------------------------------------- %
    \bigbreak % ------------------------------------------------------------------------ %

    \settowidth{\formulalinewidth}{0pt}

    \measuremaxwidth{\gamma\, =\, \text{unghiul țintă}}
    \measuremaxwidth{\theta\, =\, \text{unghiul giroscopului}}
    \measuremaxwidth{E_{raw}(t)\, =\, \gamma - \theta}
    \measuremaxwidth{E_{raw}(t)\, =\, \text{eroarea de la traiectorie}}
    \measuremaxwidth{E(t)\, =\, E_{raw}(t)\, =\, \text{eroarea finală}}

    \addtolength{\formulalinewidth}{2em}

    \begin{align*}
        \gamma\, &=\, \text{unghiul țintă} \\
        \theta\, &=\, \text{unghiul giroscopului} \\
        % --------------------------------------------------------------------------- %        
        \intertext{\centering\rule[2.2pt]{\formulalinewidth}{0.4pt}} 
        % --------------------------------------------------------------------------- %
        E_{raw}(t)\, &=\, \gamma - \theta \\
        E_{raw}(t)\, &=\, \text{eroarea de la traiectorie} \\
        E(t)\, &=\, E_{raw}(t)\, =\, \text{eroarea finală} \\ 
    \end{align*}

    % ------------------------------------------------------------------------ %
    % ------------------------------------------------------------------------ %
    \newpage % >>> ------------------------------------------------------- <<< % 
    % ------------------------------------------------------------------------ %
    % ------------------------------------------------------------------------ %

    \[P(t)\, =\, K_{p}\, \cdot\, E(t)\, \longrightarrow\, \text{proportional to the error}\]
    \[I(t)\, =\, K_{i}\, \cdot\, \int_{0}^{t} E(x)\, dx\, \longrightarrow\, \text{sum of past errors}\]
    \[I(t)\, =\, K_{i}\, \cdot\, \int_{0}^{t} E(x)\, dx\, \longrightarrow\, \text{total error accumulation}\]
    \[D(t)\, =\, K_{d}\, \cdot\, \frac{d}{dt} E(t)\, \longrightarrow\, \text{rate of change of the error}\] 

    \[C(t)\, =\, K_{p}\, \cdot\, E(t)\, +\, K_{i}\, \cdot\, \int_{0}^{t} E(x)\, dx\, +\, K_{d}\, \cdot\, \frac{d}{dt} E(t)\, \longrightarrow\, \text{final correction}\]

    \bigbreak % ------------------------------------------------------------------------ %
    \hrule % --------------------------------------------------------------------------- %
    \bigbreak % ------------------------------------------------------------------------ %

    \[E_{f}(t)\, =\, \alpha\, \cdot\, E_{raw}(t)\, +\, (1 - \alpha)\, \cdot\, E_{f}(t - 1)\, \longrightarrow\, \text{filtered error}\]
    \[\alpha \in [0, 1]\, \longrightarrow\, \text{constant for noise filtering}\]
    \[\rightarrow\, \text{cute little graph to demonstrate the noise filtering}\, \leftarrow\]

    \[E(t)\, =\, E_{raw}(t) \quad \textbf{or} \quad E_{f}(t)\]
    \[*E(t)\, =\, E_{raw}(t)\, =\, E_{f}(t)\, \rightarrow\, \alpha\, =\, 1\]

    \bigbreak % ------------------------------------------------------------------------ %
    \hrule % --------------------------------------------------------------------------- %
    \bigbreak % ------------------------------------------------------------------------ %

    \[\omega_{i} \in [0, 2^{20}]\, \longrightarrow\, I(t) \in (K_{i} \cdot [-\omega_{i}, +\omega_{i}])\]
    \[I(t)\, =\, K_{i}\, \cdot\, \max(-\omega_{i}, \min(+\omega_{i}, \int_{0}^{t} E(x)\, dx))\, \longrightarrow\, \text{integral windup}\]

    \bigbreak % ------------------------------------------------------------------------ %
    \hrule % --------------------------------------------------------------------------- %
    \bigbreak % ------------------------------------------------------------------------ %

    Definitions constants % start new paragraph % 
    \begin{align*}
        \omega_{variable}\, &=\, \text{upper bound for a variable}\, \longrightarrow\, \omega_{variable} \in [0, 2^{20}] \\
        \omega_{variable}\, &\rightarrow\, \text{variable} \in [-\omega_{variable}, +\omega_{variable}] \\
        K_{p}, K_{i}, K_{d}\, &=\, \text{constants used in the PID controller} \\   
        \alpha\, &=\, \text{constant used by the gyro noise filtering algorithm} \\
        \epsilon_{error}\, &=\, \text{range of acceptable errors}\, \rightarrow\, \epsilon_{error} \in [0, 10]\, \text{degrees}\\
        \epsilon_{error}\, &\rightarrow\, \textbf{acceptable}\, \Leftrightarrow\, (E(t) \in [-\epsilon_{error}, +\epsilon_{error}]) \\  
    \end{align*}
    
    % ------------------------------------------------------------------------ %
    % ------------------------------------------------------------------------ %
    \newpage % >>> ------------------------------------------------------- <<< % 
    % ------------------------------------------------------------------------ %
    % ------------------------------------------------------------------------ %

    \[P(t)\, =\, K_{p}\, \cdot\, E(t)\, \longrightarrow\, \text{proporțional cu eroarea}\]
    \[I(t)\, =\, K_{i}\, \cdot\, \int_{0}^{t} E(x)\, dx\, \longrightarrow\, \text{suma erorilor din trecut}\]
    \[I(t)\, =\, K_{i}\, \cdot\, \int_{0}^{t} E(x)\, dx\, \longrightarrow\, \text{suma tuturor erorilor}\]
    \[I(t)\, =\, K_{i}\, \cdot\, \int_{0}^{t} E(x)\, dx\, \longrightarrow\, \text{acumularea totală a erorii}\]
    \[D(t)\, =\, K_{d}\, \cdot\, \frac{d}{dt} E(t)\, \longrightarrow\, \text{rata de schimbare a erorii}\] 

    \[C(t)\, =\, K_{p}\, \cdot\, E(t)\, +\, K_{i}\, \cdot\, \int_{0}^{t} E(x)\, dx\, +\, K_{d}\, \cdot\, \frac{d}{dt} E(t)\, \longrightarrow\, \text{corectarea finală}\]

    \bigbreak % ------------------------------------------------------------------------ %
    \hrule % --------------------------------------------------------------------------- %
    \bigbreak % ------------------------------------------------------------------------ %

    \[E_{f}(t)\, =\, \alpha\, \cdot\, E_{raw}(t)\, +\, (1 - \alpha)\, \cdot\, E_{f}(t - 1)\, \longrightarrow\, \text{eroarea filtrată}\] \
    \[\alpha \in [0, 1]\, \longrightarrow\, \text{constantă pentru filtrarea erorii}\]

    \[E(t)\, =\, E_{raw}(t) \quad \textbf{sau} \quad E_{f}(t)\]
    \[*E(t)\, =\, E_{raw}(t)\, =\, E_{f}(t)\, \rightarrow\, \alpha\, =\, 1\]
    
    \bigbreak % ------------------------------------------------------------------------ %
    \hrule % --------------------------------------------------------------------------- %
    \bigbreak % ------------------------------------------------------------------------ %
    
    \[\omega_{i} \in [0, 2^{20}]\, \longrightarrow\, I(t) \in (K_{i} \cdot [-\omega_{i}, +\omega_{i}])\]
    \[I(t)\, =\, K_{i}\, \cdot\, \max(-\omega_{i}, \min(+\omega_{i}, \int_{0}^{t} E(x)\, dx))\, \longrightarrow\, \text{limitarea integralei}\]

    \bigbreak % ------------------------------------------------------------------------ %
    \hrule % --------------------------------------------------------------------------- %
    \bigbreak % ------------------------------------------------------------------------ %

    Definiții constante % start new paragraph % 
    \begin{align*}
        \omega_{variable}\, &=\, \text{limită pentru o variabilă}\, \longrightarrow\, \omega_{variable} \in [0, 2^{20}] \\
        \omega_{variable}\, &\rightarrow\, \text{variable} \in [-\omega_{variable}, +\omega_{variable}] \\
        K_{p}, K_{i}, K_{d}\, &=\, \text{constantele folosite în controlerul PID} \\   
        \alpha\, &=\, \text{constantă folosită pentru filtrarea zgomotului giroscopului} \\
        \epsilon_{error}\, &=\, \text{intervalul erorilor acceptabile}\, \rightarrow\, \epsilon_{error} \in [0, 10]\, \text{grade}\\
        \epsilon_{error}\, &\rightarrow\, \textbf{acceptabilă}\, \Leftrightarrow\, (E(t) \in [-\epsilon_{error}, +\epsilon_{error}]) \\  
    \end{align*}

    % ------------------------------------------------------------------------ %
    % ------------------------------------------------------------------------ %
    % ------------------------------------------------------------------------ %
    % ------------------------------------------------------------------------ %
    % ------------------------------------------------------------------------ %
    % ------------------------------------------------------------------------ %
    % ------------------------------------------------------------------------ %
    % ------------------------------------------------------------------------ %
    % ------------------------------------------------------------------------ %
    % ------------------------------------------------------------------------ %
    % ------------------------------------------------------------------------ %
    % ------------------------------------------------------------------------ %
    % ------------------------------------------------------------------------ %
    % ------------------------------------------------------------------------ %
    % ------------------------------------------------------------------------ %
    % ------------------------------------------------------------------------ %
    \newpage % >>> ------------------------------------------------------- <<< % 
    % ------------------------------------------------------------------------ %
    % ------------------------------------------------------------------------ %
    % ------------------------------------------------------------------------ %
    % ------------------------------------------------------------------------ %
    % ------------------------------------------------------------------------ %
    % ------------------------------------------------------------------------ %
    % ------------------------------------------------------------------------ %
    % ------------------------------------------------------------------------ %
    % ------------------------------------------------------------------------ %
    % ------------------------------------------------------------------------ %
    % ------------------------------------------------------------------------ %
    % ------------------------------------------------------------------------ %
    % ------------------------------------------------------------------------ %
    % ------------------------------------------------------------------------ %
    % ------------------------------------------------------------------------ %
    % ------------------------------------------------------------------------ %

    We will now start the formulas for the functions (transforming speeds and other things like that) 
    \bigbreak % ------------------------------------------------------------------------ %

    \begin{align*}
        c_{wheel}\, &=\, \frac{360^{\circ}}{2\pi r}\, \longrightarrow\, \text{necessary degrees to move 1 unit} \\
        c_{wheel}\, &=\, \frac{360^{\circ}}{\pi \cdot 49.5\, \text{mm}}\, \approx\, 2.314^{\circ}\, \longrightarrow\, \text{with our wheels} \\
    \end{align*}

    \begin{align*}    
        d_{unit}\, &=\, \text{distance in the unit of measurement} \\
        d_{degrees}\, &=\, d_{unit}\, \cdot\, c_{wheel}\, \longrightarrow\, \text{transformed distance} \\
    \end{align*}
    
    \bigbreak % ------------------------------------------------------------------------ %
    \hrule % --------------------------------------------------------------------------- %
    \bigbreak % ------------------------------------------------------------------------ %

    \begin{align*}
        s_{degrees}\, &=\, \textstyle\text{the turn speed with one wheel in $\frac{\text{degrees}}{\text{1 second}}$} \\
        d_{wheels}\, &=\, \text{the distance between the two wheels} \\
    \end{align*}

    \[d_{path-wheels}\, =\, \frac{2\pi\, \cdot\, d_{wheels}}{360^{\circ}}\, \longrightarrow\, \text{the distance to turn one degree}\] 
    \[d_{speed}\, =\, d_{path-wheels}\, \cdot\, s_{degrees}\, \longrightarrow\, \text{the distance to turn the desired speed}\]
    
    \[\rightarrow\, \left\langle \begin{array}{c}
        \text{now we know the necessary distance to move per second,} \\
        \text{so we need to transform $d_{speed}$ into motor speed} \\
        \text{(using the formulas from moving forwards and backwards)} \\
    \end{array}\, \right\rangle \leftarrow\]

    \[s_{wheel}\, =\, d_{speed}\, \cdot\, c_{wheel}\, \longrightarrow\, \text{the final motor speed :D}\]

    \bigbreak % ------------------------------------------------------------------------ %
    \hrule % --------------------------------------------------------------------------- %
    \bigbreak % ------------------------------------------------------------------------ %

    \begin{align*}
        \text{motor.stalled()}\, &=\, \textbf{true}\, \Leftrightarrow\, (\text{motor.position()} - \text{lastt}_{\text{position}} < \text{threshold}) \\
        \text{motor.stalled()}\, &=\, \text{check if a motor has stalled - can't move anymore} \\
        \text{lastt}_{\text{position}}\, &=\, \text{motor.position()}\, \rightarrow\, \text{updated and checked after a set time} \\
    \end{align*}
    \[\rightarrow\, \text{check if a motor is stalled}\, \leftarrow\]
 
    \bigbreak % ------------------------------------------------------------------------ %
    \hrule % --------------------------------------------------------------------------- %
    \bigbreak % ------------------------------------------------------------------------ %

    % ------------------------------------------------------------------------ %
    % ------------------------------------------------------------------------ %
    \newpage % >>> ------------------------------------------------------- <<< % 
    % ------------------------------------------------------------------------ %
    % ------------------------------------------------------------------------ %

    \begin{align*}
        c_{wheel}\, &=\, \frac{360^{\circ}}{2\pi r}\, \longrightarrow\, \text{gradele necesare pentru a ne mișca o unitate} \\
        c_{wheel}\, &=\, \frac{360^{\circ}}{\pi \cdot 49.5\, \text{mm}}\, \approx\, 2.314^{\circ}\, \longrightarrow\, \text{pentru roțile noastre} \\
    \end{align*}

    \begin{align*}    
        d_{unit}\, &=\, \text{distanța in unitatea de masură} \\
        d_{degrees}\, &=\, d_{unit}\, \cdot\, c_{wheel}\, \longrightarrow\, \text{distanța transformată} \\
    \end{align*}
    
    \bigbreak % ------------------------------------------------------------------------ %
    \hrule % --------------------------------------------------------------------------- %
    \bigbreak % ------------------------------------------------------------------------ %

    \begin{align*}
        s_{degrees}\, &=\, \textstyle\text{viteza de rotație cu o roată în $\frac{\text{grade}}{\text{1 secundă}}$} \\
        d_{wheels}\, &=\, \text{distanța între cele două roți}  \\
    \end{align*}

    \[d_{path-wheels}\, =\, \frac{2\pi\, \cdot\, d_{wheels}}{360^{\circ}}\, \longrightarrow\, \text{distanța pentru a roti robotul cu $1^{\circ}$}\] 
    \[d_{speed}\, =\, d_{path-wheels}\, \cdot\, s_{degrees}\, \longrightarrow\, \text{distanța pentru a ne roti cu viteza dorită}\]
    
    % bla bla bla %
    \[\rightarrow\, \left\langle \begin{array}{c}
        \text{now we know the necessary distance to move per second,} \\
        \text{so we need to transform $d_{speed}$ into motor speed} \\
        \text{(using the formulas from moving forwards and backwards)} \\
    \end{array}\, \right\rangle \leftarrow\]

    \[s_{wheel}\, =\, d_{speed}\, \cdot\, c_{wheel}\, \longrightarrow\, \text{viteza finală a motorului :D}\]

    \bigbreak % ------------------------------------------------------------------------ %
    \hrule % --------------------------------------------------------------------------- %
    \bigbreak % ------------------------------------------------------------------------ %

    \begin{align*}
        \text{motor.stalled()}\, &=\, \textbf{true}\, \Leftrightarrow\, (\text{motor.position()} - \text{lastt}_{\text{position}} < \text{prag}) \\
        \text{motor.stalled()}\, &=\, \text{verifică dacă motorul s-a oprit $-$ nu se mai poate mișca} \\
        \text{lastt}_{\text{position}}\, &=\, \text{motor.position()}\, \rightarrow\, \text{schimbată după un interval de timp} \\
    \end{align*}
    \[\rightarrow\, \text{check if a motor is stalled}\, \leftarrow\]
 
    \bigbreak % ------------------------------------------------------------------------ %
    \hrule % --------------------------------------------------------------------------- %
    \bigbreak % ------------------------------------------------------------------------ %

    % --------------------------------------------------------------------------- %

\end{document}

